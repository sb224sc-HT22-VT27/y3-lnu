\documentclass{article}
\usepackage{float}
\usepackage{amsmath}
\usepackage{graphicx}

\title{2DT904 Assignment 1}
\author{Samuel Berg}
\date{December 2024}

\begin{document}

\maketitle

\tableofcontents
\newpage

\section{Transformations I}

\begin{itemize}
    \item $R_y$ creates $4\cdot4$ homogeneous rotation matrix around the Y-axis
    \item $T(x, y, z)$ creates a $4\cdot4$ homogeneous translation matrix
    \item $p = \begin{bmatrix}10 \\ 15 \\ 2 \\ 1\end{bmatrix}$
\end{itemize}

\subsection{a. Calculate $p_t = T(1, 2, 3)p$}

Using the translation matrix $T(x, y, z)$:

$$
T(1, 2, 3) = \begin{bmatrix}
1 & 0 & 0 & 1 \\
0 & 1 & 0 & 2 \\
0 & 0 & 1 & 3 \\
0 & 0 & 0 & 1
\end{bmatrix}
$$

Calculate $p_t$:

$$
p_t = T(1, 2, 3) \cdot p =
\begin{bmatrix}
1 & 0 & 0 & 1 \\
0 & 1 & 0 & 2 \\
0 & 0 & 1 & 3 \\
0 & 0 & 0 & 1
\end{bmatrix}
\cdot
\begin{bmatrix}
10 \\ 15 \\ 2 \\ 1
\end{bmatrix}
=
\begin{bmatrix}
11 \\ 17 \\ 5 \\ 1
\end{bmatrix}
$$

Answer: $p_t = \begin{bmatrix} 11 \\ 17 \\ 5 \\ 1 \end{bmatrix}$

\subsection{b. Calculate $p_r = R_y(30^\circ)p$}

The rotation matrix $R_y(\theta)$:

$$
R_y(30^\circ) =
\begin{bmatrix}
\cos(30^\circ) & 0 & \sin(30^\circ) & 0 \\
0 & 1 & 0 & 0 \\
-\sin(30^\circ) & 0 & \cos(30^\circ) & 0 \\
0 & 0 & 0 & 1
\end{bmatrix}
=
\begin{bmatrix}
\sqrt{3}/2 & 0 & 1/2 & 0 \\
0 & 1 & 0 & 0 \\
-1/2 & 0 & \sqrt{3}/2 & 0 \\
0 & 0 & 0 & 1
\end{bmatrix}
$$

Calculate $p_r$:

$$
p_r = R_y(30^\circ) \cdot p =
\begin{bmatrix}
\sqrt{3}/2 & 0 & 1/2 & 0 \\
0 & 1 & 0 & 0 \\
-1/2 & 0 & \sqrt{3}/2 & 0 \\
0 & 0 & 0 & 1
\end{bmatrix}
\cdot
\begin{bmatrix}
10 \\ 15 \\ 2 \\ 1
\end{bmatrix}
=
\begin{bmatrix}
5\sqrt{3} + 1 \\ 15 \\ -5 + 2\sqrt{3} \\ 1
\end{bmatrix}
$$

Answer: $p_r = \begin{bmatrix} 5\sqrt{3} + 1 \\ 15 \\ -5 + 2\sqrt{3} \\ 1 \end{bmatrix}$

\subsection{c. Effect of $T(10, 0, 10)R_y(90^\circ)T(-10, 0, -10)$}

\begin{enumerate}
    \item \textbf{Translation} $T(-10, 0, -10)$: Moves the object to position $(-10, 0, -10)$.
    \item \textbf{Rotation} $R_y(90^\circ)$: Rotates the object $90^\circ$ around the Y-axis.
    \item \textbf{Translation} $T(10, 0, 10)$: Moves the object back to position $(10, 0, 10)$.
\end{enumerate}
\textbf{Effect}: The transformation applies a $90^\circ$ rotation around the Y-axis about the point $(10, 0, 10)$.

\newpage
\section{Transformations II}

Apply $R(45^\circ)T(3, 1)$ on the house drawing.
\begin{itemize}
    \item \textbf{Translation} $T(3, 1)$: Moves the house 3 units right and 1 unit up.
    \item \textbf{Rotation} $ R(45^\circ) $: Rotates the house $45^\circ$ counterclockwise around the origin.
\end{itemize}

\textbf{Drawing:}

\begin{figure}[H]
    \centering
    \includegraphics[width=0.5\textwidth]{./img/house_transformation.png}
    \caption{Rotated house}
    \label{fig:house}
\end{figure}

\newpage
\section{Per-triangle data}

\subsection{a. Implications for the host data structure of the mesh:}
\begin{itemize}
    \item Data must explicitly store triangles instead of shared vertices.
    \item Each triangle may duplicate vertex information, increasing memory usage.
\end{itemize}

\subsection{b. Data to upload to the GPU:}
\begin{itemize}
    \item The GPU requires all vertex data for each triangle, including normals and texture coordinates.
    \item This increases bandwidth requirements compared to per-vertex data.
\end{itemize}

\newpage
\section{Visibility determination}

\subsection{a. Z-buffer}

\begin{itemize}
    \item \textbf{How it works:} Tracks depth for each pixel in a frame buffer. A pixel is updated only if the new fragment is closer.
    \item \textbf{Pipeline location:} Rasterization stage.
    \item \textbf{Data:} Depth values.
    \item \textbf{Limitations:} Does not resolve transparency.
\end{itemize}

\subsection{b. Frustum culling}

\begin{itemize}
    \item \textbf{How it works:} Discards objects outside the camera view frustum.
    \item \textbf{Pipeline location:} Before rasterization.
    \item \textbf{Data:} Bounding boxes.
    \item \textbf{Limitations:} May cull objects partially visible.
\end{itemize}

\subsection{c. Backface culling}

\begin{itemize}
    \item \textbf{How it works:} Removes triangles facing away from the camera by checking normal direction.
    \item \textbf{Pipeline location:} Before fragment shading.
    \item \textbf{Data:} Triangle normals.
    \item \textbf{Limitations:} Doesn’t work for transparent objects.
\end{itemize}

\newpage
\section{Mapping to pixels}

\subsection{a. Two crucial transformations:}

\begin{enumerate}
    \item \textbf{View transformation}
    \item \textbf{Projection transformation}
\end{enumerate}

\subsection{b. Explanation of transformations:}

\begin{enumerate}
    \item \textbf{View transformation:} Aligns the 3D world to the camera’s coordinate system.
    \item \textbf{Projection transformation:} Maps 3D points to a 2D plane, converting depth into a scalar value.
\end{enumerate}

\end{document}
